\documentclass[a4paper,12pt]{article}

\usepackage[utf8]{inputenc}   % Encodage UTF-8
\usepackage[T1]{fontenc}      % Bon encodage des caractères pour le PDF
\usepackage[french]{babel}    % Langue française
\usepackage{lmodern}          % Police de caractères
\usepackage{geometry} 
\usepackage{amsmath}        % Marges
\geometry{margin=2.5cm}

\title{draft}   % Titre (facultatif)
\author{Lancelot Ravier}  % Auteur (facultatif)
\date{\today}    % Date (facultatif)

\begin{document}

\maketitle

\section{Introduction}

Le projet expedition5300 a pour but d'analyser les effets de la vie en très haute altitude et les effets sur les habitants. Des recherches ont déjà été effectuées sur les adultes mais pas sur les enfants. Ici, on aura donc des données sur des enfants selon deux tranches d'age.

\begin{itemize}
    \item Etablir le risque, les conséquences et les
    marqueurs de l’anémie et de la déficience en fer chez l’enfant
    d’altitude.
    \item Etudier le développement de la masse
    d’hémoglobine et des volumes sanguins, de la viscosité sanguine et
    du système cardiovasculaire chez l’enfant d’altitude.
\end{itemize}

Hemoglobine : proteine permettant a l'oxygene de s'attacher au globule rouge
Hematocrite : taux de globule rouge dans le sang
\section{Données}

Les données ont été recueuillies suivant plusieurs expeditions (2), par plusieurs personnes différentes. En conséquent, il y'a plusieurs fichiers de données à rassembler. Un travail a été fait en amont pour traiter les outliers donc pas besoin de porter d'analyses dessus. En ce qui concerne les différentes bases de données, elles possèdent les mêmes clé primaires donc il est possible de retrouver un patient dans les différentes bases de donneés par son code annonymisé de la manière suivante : 

$$ id = \text{2 premières lettres du prénom} + \text{clé unique} + \text{2 premières lettres du nom}$$

\subsection{Préparation des données}

On peut tout d'abord voir que les bases de donneés ne sont pas de la meme taille :

\begin{itemize}
    \item Anemie, Cardio, nfs\_visco : 635
    \item template : 638
\end{itemize}

Il y'a donc 3 patients qui ne sont présents que dans la base de données template (BEJ0674, 640, QUJI740)

\begin{itemize}
    \item Le 640 est surement quelqu'un qui s'est desisté avant de venir ?
    \item On a bien toutes les informations pour BEJO674 et QUJI740 (peut etre des gens qui se sont desistés ou qui se sont présenté deux fois)
\end{itemize}

Afin de continuer, il faut supprimer la ligne 638. Aussi, on supprimera les lignes appartenant aux deux autres identifiants vù au dessus.

\end{document}

Pour tout, mediane et moyenne interquartile a disposition (faire des tests parametrique et non parametrique)

Se renseigner sur si on doit utiliser des percentile ou des ecarts-types (reference interval regarder dans le papier en anglais) pour définir nos propres seuils (rgarder si ça a la meme signification ou pas pour définir des valeurs normales)? 

Faire une comparaison entre Hemoglobine et Hbmass regarder sur le papier de M1 pour voir la question a répondre elle a dea ete repindue.

HBMASS que chez les 8-12 ans

Kappa de Cohen pour comparer les métriques ?

Graphiues de discordances entre deux tests

Ne pas regarder les 4 villes en meme temps car 

Retrouver les résultats de titouan !
Comparer percentile ou ecart-type ??

Mettre une colonne normale ou non